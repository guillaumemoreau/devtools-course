\subsection{Boost Test}
\label{sec:boost:test}


\begin{frame}{Boost ?}
\begin{itemize}
\item Boost est un ensemble de bibliothèques C++
\begin{itemize}
\item objectif "productivité"
\end{itemize}
\item Des compléments de la bibliothèque standard
\begin{itemize}
\item quelques interactions avec la norme
\item i.e. Boost préfigure la norme
\end{itemize}
\item Plus facile à intégrer que Google Test : peut s'utiliser uniquement avec des \texttt{.h} seuls
\item Meilleur support du multi-plateforme
\item facile à utiliser via cmake comme Visual Studio
\end{itemize}
\end{frame}

\begin{frame}{Boost.Test}
\begin{itemize}
\item La partie de Boost dédiée aux tests
\item Utilisable uniquement en \texttt{.h} même si ce n'est pas recommandé
\item basée sur un ensemble de macros (répétitives)
\item Courbe d'apprentissage simple
\item Test cases et Tests suites
\item tests paramétriques, fixtures, exceptions, death tests, gestion des flottants
\item différentes formes de sorties (texte, xml...), intégration avec d'autres outils
\item Multi-plateformes
\end{itemize}
\end{frame}

\begin{frame}[fragile]
\frametitle{Au plus simple !}
\begin{lstlisting}
#define BOOST_TEST_MODULE My Test
#include <boost/test/included/unit_test.hpp>

BOOST_AUTO_TEST_CASE(first_test)
{
  int i = 1;
  BOOST_TEST(i);
  BOOST_TEST(i == 2);
}
\end{lstlisting}
\begin{itemize}
\item \texttt{BOOST\_TEST\_MODULE} le nom du module de test
\item Utiliser le répertoire \texttt{included} pour une version utilisation uniquement les \texttt{.h}
\item \texttt{BOOST\_AUTO\_CASE\_TEST} définition d'un cas de test
\item \texttt{BOOST\_TEST} : test un prédicat
\end{itemize}
\end{frame}

\begin{frame}[fragile]
\frametitle{Compilation et exécution}
\begin{itemize}
\item Compilation : il  suffit d'ajouter le répertoire de boost aux répertoires d'include
\item Version de boost utilisée : 1.71.0
\begin{verbatim}
g++ -o boost1 -I/Users/moreau/dev/local/boost boost1.cpp
\end{verbatim}
\item Exécution en ligne de commande
\begin{lstlisting}[language=Bash]
./boost1

Running 1 test case...
boost1.cpp:8: error: in "first_test": check i == 2 has failed [1 != 2]

*** 1 failure is detected in the test module "My Test"

\end{lstlisting}
\end{itemize}
\end{frame}

\begin{frame}[fragile]
\frametitle{Test d'une fonction factorielle}
\begin{itemize}
\item \texttt{factorielle.cpp}
\begin{lstlisting}
int factorielle(int x) {
  if (x==1) {
    return 1;
  }
  else {
    return x*factorielle(x-1);
  }
}
\end{lstlisting}
\item \texttt{factorielle.h}
\begin{lstlisting}
int factorielle(int x);
\end{lstlisting}
\end{itemize}
\end{frame}

\begin{frame}[fragile]
\frametitle{Code de test}
\begin{itemize}
\item \texttt{boost2.cpp}
\begin{lstlisting}
#define BOOST_TEST_MODULE Mes fonctions
#include <boost/test/included/unit_test.hpp>

#include "factorielle.h"

BOOST_AUTO_TEST_CASE(GereLesValeursPositives)
{
  BOOST_CHECK(factorielle(5) == 120);
  BOOST_CHECK(factorielle(1) == 1);
  BOOST_CHECK_EQUAL(factorielle(1),1);

  // BOOST_CHECK(factorielle(4) == 13);
  BOOST_CHECK_EQUAL(factorielle(4),24);
}

BOOST_AUTO_TEST_CASE(GereLeZero)
{
  BOOST_CHECK(factorielle(0) == 1);

}
\end{lstlisting}
\item \texttt{BOOST\_CHECK\_EQUAL()} est plus intéressant que \texttt{BOOST\_CHECK} (en cas d'erreur)
\end{itemize}
\end{frame}

\begin{frame}[fragile]
\frametitle{Compilation et résultats}
\begin{itemize}
\item Compilation
\begin{lstlisting}[language=Bash]
g++ -o boost2 -I/Users/moreau/dev/local/boost/ boost2.cpp factorielle.cpp
\end{lstlisting}
\item Résultat
\begin{lstlisting}[language=Bash]
./boost2

Running 3 test cases...
unknown location:0: fatal error: in "GereLeZero": memory access violation at address: 0x7ffeea63cff8: invalid permissions
boost2.cpp:23: last checkpoint

*** 1 failure is detected in the test module "Mes fonctions"
\end{lstlisting}
\pause \item Beaucoup plus intéressant !
\end{itemize}
\end{frame}

\begin{frame}{Les macros Boost}
\begin{itemize}
\item De la forme \texttt{BOOST\_level[\_test](...)}
\item où \texttt{level} vaut
\begin{itemize}
\item \texttt{WARN} affichera alors un avertissement
\item \texttt{CHECK} affichera une erreur
\item \texttt{REQUIRE} affichera une erreur fatale et arrêtera l'exécution du test
\end{itemize}
\item et \texttt{test} peut être :
\begin{itemize}
\item rien : vérifiera ce qui est entre les parenthèses
\item \texttt{EQUAL}, \texttt{GE}, \texttt{GT}, \texttt{LE}, \texttt{LT}, \texttt{NE}
\item \texttt{CLOSE(left,right,tolerance)} pour les égalités entre flottants
\item \texttt{SMALL(value,tolerance)} pour les valeurs $\epsilon$
\item \texttt{EQUAL\_COLLECTIONS(l\_begin,l\_end,r\_begin,r\_end)} pour tester l'égalité de collections
\item on peut aussi vérifier la levée d'exceptions, les prédicats (foncteurs) et pas mal d'autres choses
\end{itemize}
\end{itemize}
\end{frame}

\begin{frame}[fragile]
\frametitle{Contexte de test : les fixtures}
\begin{itemize}
\item Pour mener des tests on a souvent besoin de données initiale
\item Dans Boost, il faut créer des \texttt{struct} !
\item Ici on suppose disposer d'une classe Pile classique
\item Création de la structure d'initialisation des données
\begin{lstlisting}
struct MesDonnees {
  pile p;
  pile vide;

  MesDonnees() {
      p.push(2);
      p.push(3);
      p.push(4);
  }

  ~MesDonnees() {
  }
};
\end{lstlisting}
\end{itemize}
\end{frame}

\begin{frame}[fragile]
\frametitle{Contexte de test : les fixtures}
\begin{itemize}
\item Utilisation de la fixture
\begin{lstlisting}
BOOST_FIXTURE_TEST_SUITE(Piles,MesDonnees)

BOOST_AUTO_TEST_CASE(PileVide) {
  BOOST_CHECK_EQUAL(vide.size(),0);
}

BOOST_AUTO_TEST_CASE(PileStandard) {
  BOOST_CHECK_EQUAL(4,p.top());
  BOOST_CHECK_EQUAL(3,p.size());
}
BOOST_AUTO_TEST_SUITE_END()
\end{lstlisting}
\item Résultats
\begin{lstlisting}[language=Bash]
./boost3

Running 2 test cases...

*** No errors detected
\end{lstlisting}
\item ici on a créé une \textit{Test Suite}, un ensemble de cas de tests
\end{itemize}
\end{frame}

\begin{frame}[fragile]
\frametitle{Intégration (simple) avec cmake}
\begin{itemize}
\item On reprend l'exemple de la pile + fixtures
\item Création d'une arborescence
	\begin{itemize}
	\item \texttt{src} pour les fichiers source
	\item \texttt{test} pour les fichiers de test
	\end{itemize}
\item \texttt{CMakeLists.txt} minimal pour construire un exécutable
\begin{lstlisting}[language=Bash]
cmake_minimum_required(VERSION 2.8.4)

project(boost_test_cmake)

set(SOURCE_FILES src/main.cpp)

add_executable(main_exe ${SOURCE_FILES})
\end{lstlisting}
\item Compilation et exécution
\begin{lstlisting}[language=Bash]
mkdir build
cd build
cmake ..
make
./main_exe
\end{lstlisting}

\end{itemize}
\end{frame}

\begin{frame}[fragile]
\frametitle{Ajout des tests}
\begin{itemize}
\item les tests dans le répertoires test
\begin{lstlisting}[language=Bash]
set(TEST_FILES test/fixtures.cpp)
\end{lstlisting}
\item Ajout de boost à la main (version non compilée)
\begin{lstlisting}[language=Bash]
set(BOOST_DIR /Users/moreau/dev/local/boost)
\end{lstlisting}
\item Répertoires pour les include
\begin{lstlisting}[language=Bash]
include_directories(src ${BOOST_DIR})
\end{lstlisting}
\item Outils pour le test
\begin{lstlisting}[language=Bash]
enable_testing()
add_executable(test_exe ${TEST_FILES})
add_test(Test test_exe)
\end{lstlisting}
\item relancer \texttt{cmake}, \texttt{make} puis lancer les tests avec \texttt{ctest}
\end{itemize}
\end{frame}

\begin{frame}[fragile]
\frametitle{Résultats}
\begin{lstlisting}[language=Bash]
ctest

Test project /Users/moreau/Documents/Enseignement/optionRV/MEDEV/tests/code/boost/build
    Start 1: Test
1/1 Test #1: Test .............................   Passed    0.01 sec

100% tests passed, 0 tests failed out of 1

Total Test time (real) =   0.01 sec
\end{lstlisting}

\end{frame}

\begin{frame}[fragile]
\frametitle{Pour aller plus loin : plusieurs fichiers pour les tests}
\begin{itemize}
\item Pas envie de mettre tous ses tests dans le même fichier
\item Exemple simple : on réintègre \texttt{boost2.cpp} dans test
\item Quelques modifications dans le \texttt{CMakeLists.txt}
\begin{lstlisting}[language=Bash]
set(TEST_FILES test/fixtures.cpp test/boost2.cpp)

add_executable(tf_exe boost2.cpp fonctions.cpp)
add_test(TextFact tf_exe)
\end{lstlisting}
\item On peut ne faire qu'un seul exécutable de test bien sûr, mais alors un seul \texttt{BOOST\_TEST\_MODULE} (il contient un main)
\end{itemize}
\end{frame}

\begin{frame}[fragile]
\frametitle{Résultats}
\begin{lstlisting}[language=Bash]
ctest

Test project /Users/moreau/Documents/Enseignement/optionRV/MEDEV/tests/code/boost/build
    Start 1: TestFixtures
1/2 Test #1: TestFixtures .....................   Passed    0.01 sec
    Start 2: TextFact
2/2 Test #2: TextFact .........................***Failed    0.05 sec

50% tests passed, 1 tests failed out of 2

Total Test time (real) =   0.06 sec

The following tests FAILED:
	  2 - TextFact (Failed)
Errors while running CTest
\end{lstlisting}

\end{frame}
