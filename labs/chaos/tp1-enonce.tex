% source : http://codegolf.stackexchange.com/questions/35328/3d-discrete-dogfighting-duel-now-open-to-non-java-submissions?lq=1


\section{Bataille aérienne}\label{bataille-auxe9rienne}

Le but de ce TP est simuler une bataille aérienne entre 2 escadrilles,
la vôtre (i.e.~celle que vous aurez programmée avec votre tactique) et
celle contrôlée par l'ordinateur. A chaque, les avions pourront bouger
et tirer. Très simple \ldots{} en apparence.

\subsection{Terrain de jeu}\label{terrain-de-jeu}

Pour ne pas trop compliquer le TP, le combat se déroule dans un cube
discret de côté \emph{n} (par exemple, on peut prendre n=14) dans lequel
se trouvent vos 2 avions et les 2 avions de l'ordinateur. Vos avions
partent des positions (0,5,0) et (0,8,0), ceux de l'ordinateur de
(13,5,13) et (13,8,13). Tous les avions commencent le jeu en vol
horizontal et en s'éloignant du mur vertical dont ils sont le plus
proches.

\subsection{Mouvements}\label{mouvements}

Comme ce sont des avions et non des hélicoptères, il y a quelques règles à
respecter : il n'est pas possible de changer instantanément de direction
et encore moins de rester sur place. L'avion doit conserver une sorte de
``cap'' et avancer dans la direction de ce cap. A chaque tour de jeu, il
ne peut changer de direction qu'au maximum de 45 degrés autour de ce
cap. Dans un tour de jeu, tous les avions annoncent à tous les autres
leur nouveau cap puis effectuent leur mouvement. Deux avions qui
arrivent sur la même case à un même pas de temps sont tous les deux détruits.

Attention, les avions de l'ordinateur peuvent tricher : ils peuvent traverser les
murs (et réapparaitre en face). Pour le reste, ils essaient toujours de
s'aligner vers un vos deux vos avions (le plus proche d'eux à l'instant) pour
vous tirer dessus. Vous pourrez organiser la tactique de vos avions
comme vous le souhaitez : soit ils agiront de façon individuelle, soit
coordonnée. Attention, vos avions n'ont pas le droit de traverser les
murs. Si un avion heurte un mur, il est détruit.

\subsection{Tir et destruction}\label{tir-et-destruction}

Chaque avion ne peut tirer qu'une fois au plus par tour et ceci doit
être décidé en même temps que la direction de vol. Le tir se fait bien
entendu dans la direction de l'avion et a lieu juste après le mouvement
de l'avion. Après chaque tir, il faut refroidir le canon pendant 1
tour. Le canon fonctionne de façon instantanée et détruit immédiatement
l'avion se trouvant sur sa trajectoire. Il ne peut détruire qu'un avion
à la fois. Deux avions peuvent se tirer dessus en même temps et donc se
détruire tous les deux.

Pour que le jeu se termine, on peut considérer que le canon ``tire
large'', c'est-à-dire dans le cas où le canon tire sur une ligne
horizontale, les lignes à gauche, droite, supérieures et inférieures
font partie de la zone où le canon fera des dégâts.

\section{Travail à effectuer}\label{travail-uxe0-effectuer}

Il est attendu \textbf{un seul TP pour toute l'option} : vous devez vous
organiser pour écrire le code pour les deux types d'avions, la zone de
combat, la visualisation en 3D (OpenGl/GLUT ou OpenSceneGraph). Bien sûr
vous pouvez écrire plusieurs stratégies différentes pour vos avions. Celles-ci peuvent individuelles ou coordonnées, comporter des parts offensives et défensives...


